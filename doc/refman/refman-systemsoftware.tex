\chapter{System Software}
\label{chap:systemsoftware}

\section{System Libraries}

The system libraries are build in a unified environment, that means
all libraries are build together and managed by one build system. When
checking out an empty directory, the clean directory structure at base
level (that is \verb|src/sw/system/dm|) looks like:

\begin{lstlisting}
apps/
autogen.sh
bootrom/
configure.ac
doc/
m4/
Makefile.am
src/
\end{lstlisting}

The grey folders and files are part of the (autotools) build
structure. The libraries can be found in \verb|src| and will be
discussed below. The \verb|apps| folder contains (example)
applications that are shipped with the distribution. They will
actually not be build in place but together with a system
configuration in your system's environment.

The \verb|doc| folder contains the Doxygen configuration to build the
API documentation (pdf and html) you can also find online. Finally,
\verb|bootrom| contains the bootcode and the necessary programs to
translate the binary to verilog code to include in the hardware.

\section{Building and Installing the System Libraries}

The system software build system is based on autotools (autoconf,
automake) and libtool. Make sure you have the necessary tools
installed.

In the current system state we encourage you to build and install the
system libraries in place. Most of the descriptions and tutorial parts
relate to that scheme. When doing so (as described below) you will
install libraries, headers and scripts needed for building software in
the system software folder in \verb|lib/|, \verb|include/| and
\verb|share/|.

\subsection{Generate the \texttt{configure} script}

Although the autotools are complex, generating the \verb|configure|
script is easily done running the \verb|autogen.sh| script. Just run
the command in place:

\begin{lstlisting}
./autogen.sh
\end{lstlisting}

Afterwards, the \verb|configure| script will be present. As mentioned
above, you should not run it in place but create a separate building
directory:

\begin{lstlisting}
mkdir build
cd build
\end{lstlisting}

The \verb|autogen.sh| script already gave you the commonly used
configure command line:

\begin{lstlisting}
../configure CFLAGS='-g -O0' --prefix=`pwd`/.. --host=or32 \
  CC=or32-elf-gcc
\end{lstlisting}

This assumes you have an environment variable \verb|OPTIMSOC_NEWLIB|
set. Alternatively you can set the path using
\verb|--with-optimsoc-newlib|. You can also change the compiler if
necessary. The Makefile templates for libraries are also built based
on the \verb|CC| setting. As they also contain calls to \verb|objdump|
and \verb|objcopy|, the script tries to replace gcc in the \verb|CC|
variable with those values. In case the programs are called different
you can set \verb|OR32OBJDUMP| and \verb|OR32OBJCOPY| environment
variables in the configure run accordingly.

\subsection{Build and Install}

Afterwards you are ready to build your software

\begin{lstlisting}
make
\end{lstlisting}

and then install it

\begin{lstlisting}
make install
\end{lstlisting}

Afterwards you will find the libraries in \verb|$(prefix)/lib| (in
most cases directly in the system software base folder). Furthermore
the headers plus extra build files are installed.

\subsection{Develop System Libraries}

Development

\section{Building Software}