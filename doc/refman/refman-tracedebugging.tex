\chapter{Trace Debugging}

\section{Software Trace}

The most convenient way to debug a system are software traces. A
software trace is a stream of events which are mainly triggered by the
software. A trace event is defined by an \verb|index| and a
\verb|value|. Complex events can be defined as sequences of single
events (see discussion below).

\subsection{Software Instrumentation}

The OpenRISC architecture defines the no operation instruction
(\verb|l.nop|) with a 24-bit operand. Inspired by the OpenRISC
testbenches and simulator, we use this minimally intrusive way to
instrument the source code. A trace event can be defined in assembly
as

\begin{alltt}
l.nop \emph{<index>}
\end{alltt}

The \verb|value| parameter is handled with general purpose register 3.
The current value at the \verb|nop| execution is emitted. So, three
instructions are required:

\begin{alltt}
l.movhi r3, hi(\emph{<value>})
l.ori r3, r3, lo(\emph{<value>})
l.nop \emph{<index>}
\end{alltt}

For convenience a macro is defined for use in C code:

\begin{alltt}
OPTIMSOC_TRACE(\emph{<index>}, \emph{<value})
\end{alltt}

The C library and the OpTiMSoC libraries use trace events to ease
debugging, the debugging infrastructure allows to trace the events or
filter them. The debugging events are grouped and one group of indexes
(\verb|0x1500| and above) are defined for users.

\subsubsection{Basic Trace Events}

\paragraph{Exit (0x1)} The library has terminated (either main
returned or exit has been called). The value contains the exit code as
returned by main or passed to the exit call.

\paragraph{Printf (0x4)} Characters are printed directly from the
library by using sequences of the trace events. Each of the events
contains a character.

\paragraph{Exception (0x10)} An exception occured (program counter
observed). The exception vector number is given as value (1: Reset, 2:
Bus fault, 3: Data page fault, 4: Instruction page fault, 5: Timer, 6:
Alignment Error, 7: Illegal instruction, 8: Interrupt, 9: Data TLB
miss, 10: Instruction TLB miss, 11: Range Exception, 12: System call)

\paragraph{Return from exception (0x11)} The \verb|l.rfe| instruction
was executed.

\subsubsection{Trace Sections}

For display in the user interface and to distinct difference phases of
an applications execution (such as threads), sections can be used. A
section is first defined by a name. Later a section is entered with a
trace event which terminates the previous section.

\paragraph{Start Section Definition (0x20)} Start the definition of
the section identified by \verb|value|.

\paragraph{Section Definition (0x21)} A sequence of characters that
identify the secquence.

\paragraph{Section Enter (0x22)} Enter the section with identifier
\verb|<value>|.

\paragraph{Activate Exceptions (0x23)} Trigger this event to preserve
sections during exceptions (in user interface). When an exception
occurs, this is accordingly displayed and after the exception ends,
the previous section is restored.

\paragraph{Global Section Definition (0x25)} Used instead of 0x21 to
define a section globally.


