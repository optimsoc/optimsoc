\chapter{Installation \& Configuration}
\label{chap:installation}

Installing an OpTiMSoC release should not take more than 10~minutes (if you have a decent internet connection), so let's get started!

\section{System Requirements}
Please first check the system requirements.

\begin{itemize}
 \item Supported Linux distribution: Ubuntu 14.04 or 16.04 LTS on x86\_64
 \item 20 GB disk space (mostly for Xilinx Vivado and other tools)
 \item 4~GB or more of RAM helps greatly (especially during FPGA synthesis)
 \item We recommend not using a VM, but running directly on the hardware.
   Using a VM is possible, but will result in significantly slower compilation and synthesis runs.
 \item (optional, only if you want to do FPGA synthesis) Xilinx Vivado 2016.2 (the free WebPack edition is sufficient for most use cases)
\end{itemize}

\begin{docnote}
Some external tools, especially EDA tools for synthesis and simulators, might be required that are in some cases proprietary and cost some money. Although OpTiMSoC is developed at an university with access to many EDA tools, we aim to always provide tool flows and support for open and free tools, but especially when it comes to synthesis such alternatives are even not available.
\end{docnote}

\section{Dependencies}
In OpTiMSoC, we try to use only external dependencies which are known to be stable and readily available.
Some can be installed as distribution packages, and others can be downloaded as binaries from us.

First, install all required packages from Ubuntu.

\begin{lstlisting}[language=sh]
sudo apt-get -y install curl git build-essential tcl libusb-1.0-0-dev \
  libboost-dev libelf-dev swig python3 python3-yaml

# optional, but highly recommended: a waveform viewer
sudo apt-get -y install gtkwave
\end{lstlisting}

Additionally, we need two things which are not available as Ubuntu packages right now: a recent version of Verilator, and the or1k-elf-multicore toolchain (compiler, C library, debugger, etc.).
Install them with our binary installation script:

\begin{lstlisting}[language=sh]
# if it does not exist yet: prepare the /opt/optimsoc directory
sudo mkdir /opt/optimsoc
sudo chown $USER /opt/optimsoc

# download and install the prebuilt tools
curl -O https://raw.githubusercontent.com/optimsoc/prebuilts/master/optimsoc-prebuilt-deploy.py
sudo python optimsoc-prebuilt-deploy.py -d /opt/optimsoc verilator or1kelf
\end{lstlisting}

To use the prebuilt tools some environment variables need to be adjusted.
This is done by running the following command \textbf{in every terminal session that you want to use OpTiMSoC in}:

\begin{lstlisting}[language=sh]
source /opt/optimsoc/setup_prebuilt.sh
\end{lstlisting}

\begin{docnote}
Automatically load the prebuilts in every new terminal session by adding it to your \verb|~/.bashrc| file:

\begin{lstlisting}[language=sh]
echo 'source /opt/optimsoc/setup_prebuilt.sh' >> ~/.bashrc
\end{lstlisting}
\end{docnote}


\section{Install OpTiMSoC}
Now that all preparations are done, you are now ready to install the OpTiMSoC framework itself.
There are two options: either, you can install a prebuilt release package, or you can build OpTiMSoC yourself from the sources.
We recommend starting with a binary release installation, and move to a custom-built version only after you verified that everything works.

\subsection{Recommended: OpTiMSoC binary releases}
The most simple way to get started is with the release packages. You
can find the OpTiMSoC releases here:
\url{https://github.com/optimsoc/sources/releases}. With the release
you can find the distribution packages that can be extracted into any
directory and used directly from there. The recommended default is to
install OpTiMSoC into \verb|/opt/optimsoc|. There are two packages:
the \verb|base| package contains the programs, libraries and tools
to get started. The \verb|examples| package contains prebuilt example systems (both in simulation and FPGA bitstreams) for the real quick start.

To install the 2016.1 release into \verb|/opt/optimsoc|, run the following commands:

\begin{lstlisting}[language=sh]
wget https://github.com/optimsoc/sources/releases/download/v2016.1/optimsoc-2016.1-base.tar.gz
wget https://github.com/optimsoc/sources/releases/download/v2016.1/optimsoc-2016.1-examples.tar.gz
tar -xf optimsoc-2016.1-base.tar.gz -C /opt/optimsoc
tar -xf optimsoc-2016.1-examples.tar.gz -C /opt/optimsoc
\end{lstlisting}

To use OpTiMSoC multiple environment variables need to be set.
This is done by running the following command \textbf{in every terminal session that you want to use OpTiMSoC in}:

\begin{lstlisting}[language=sh]
cd /opt/optimsoc/2016.1
source optimsoc-environment.sh
\end{lstlisting}

\begin{docnote}
Automatically load the OpTiMSoC environment in every new terminal session by adding it to your \verb|~/.bashrc| file:

\begin{lstlisting}[language=sh]
echo 'cd /opt/optimsoc/2016.1; source optimsoc-environment.sh' >> ~/.bashrc
\end{lstlisting}
\end{docnote}

\medskip
Installation complete!
You are now ready to go to the tutorials in Chapter~\ref{chap:tutorials}.


\subsection{Alternative: Build OpTiMSoC from sources}

You can also build OpTiMSoC from the sources. This options usually
becomes standard if you start developing for or around OpTiMSoC. The
build is done from one git repository:
\url{https://github.com/optimsoc/sources}.

It is generally a good idea to understand git, especially when you
plan to contribute to OpTiMSoC. Nevertheless, we will give a more
detailed explanation of how to get your personal copies of OpTiMSoC
and (potentially) update them.

First, you need some additional tools (the ``build dependencies''):
\begin{lstlisting}[language=sh]
sudo apt-get -y install texlive texlive-latex-extra texlive-fonts-extra
\end{lstlisting}

Then get the sources from git:
\begin{lstlisting}[language=sh]
git clone https://github.com/optimsoc/sources.git optimsoc-sources
cd optimsoc-sources
# optional: checkout a release version
git checkout v2016.1
\end{lstlisting}

Now you're ready to build OpTiMSoC.
OpTiMSoC contains a Makefile which controls the whole build process.
Building is as simple as calling (inside the top-level source directory that you just got from git)

\begin{lstlisting}[language=sh]
make
\end{lstlisting}

By default this also builds the documentation, the Verilator examples and the FPGA bitstreams (which requires Xilinx Vivado to be working).
You can disable some features by passing variables to the \verb|Makefile|:

\begin{lstlisting}[language=sh]
# only build Verilator examples, but no bitstreams and no docs
make BUILD_EXAMPLES=yes BUILD_EXAMPLES_FPGA=no BUILD_DOCS=no
\end{lstlisting}

If you need even more fine-grained control over the build process, call the build script \verb|tools/build.py| directly.
Running \verb|tools/build.py --help| will give you a list of all available options.

After the build process, all build artifacts are located in \verb|objdir/dist|.
You can either use OpTiMSoC directly from there (good during development), or copy it to a more suitable installation location in \verb|/opt/optimsoc/VERSION| by running

\begin{lstlisting}[language=sh]
make install
\end{lstlisting}

You can also modify the target directory using environment variables passed to
\verb|make|. This is especially useful if you don't have enough permissions to
write to \verb|/opt/optimsoc|.

\begin{itemize}
  \item Use \verb|INSTALL_PREFIX| to change the installation prefix from
    \verb|/opt/optimsoc| to something else. The installation will then go
    into \verb|INSTALL_PREFIX/VERSION|.
  \item Use \verb|INSTALL_TARGET| to fully override the installation path.
    The installation will then go exactly into this directory.
\end{itemize}

\begin{lstlisting}[language=sh]
# use INSTALL_PREFIX to install into ~/optimsoc/VERSION
make install INSTALL_PREFIX=~/optimsoc

# full control for special cases: use INSTALL_TARGET
# to install into ~/optimsoc-testversion
make install INSTALL_TARGET=~/optimsoc-testversion
\end{lstlisting}


Independent of which directory you chose, to use OpTiMSoC multiple environment variables need to be set.
This is done by running the following command \textbf{in every terminal session that you want to use OpTiMSoC in}:

\begin{lstlisting}[language=sh]
cd YOUR_INSTALLATION_DIR
source optimsoc-environment.sh
\end{lstlisting}

See the binary installation description above for information on how to make this change permanent.

\medskip
OpTiMSoC is now ready to be used and you can continue with the tutorials in Chapter~\ref{chap:tutorials}.
